% Options for packages loaded elsewhere
\PassOptionsToPackage{unicode}{hyperref}
\PassOptionsToPackage{hyphens}{url}
\PassOptionsToPackage{dvipsnames,svgnames,x11names}{xcolor}
%
\documentclass[
  letterpaper,
  DIV=11,
  numbers=noendperiod]{scrreprt}

\usepackage{amsmath,amssymb}
\usepackage{iftex}
\ifPDFTeX
  \usepackage[T1]{fontenc}
  \usepackage[utf8]{inputenc}
  \usepackage{textcomp} % provide euro and other symbols
\else % if luatex or xetex
  \usepackage{unicode-math}
  \defaultfontfeatures{Scale=MatchLowercase}
  \defaultfontfeatures[\rmfamily]{Ligatures=TeX,Scale=1}
\fi
\usepackage{lmodern}
\ifPDFTeX\else  
    % xetex/luatex font selection
\fi
% Use upquote if available, for straight quotes in verbatim environments
\IfFileExists{upquote.sty}{\usepackage{upquote}}{}
\IfFileExists{microtype.sty}{% use microtype if available
  \usepackage[]{microtype}
  \UseMicrotypeSet[protrusion]{basicmath} % disable protrusion for tt fonts
}{}
\makeatletter
\@ifundefined{KOMAClassName}{% if non-KOMA class
  \IfFileExists{parskip.sty}{%
    \usepackage{parskip}
  }{% else
    \setlength{\parindent}{0pt}
    \setlength{\parskip}{6pt plus 2pt minus 1pt}}
}{% if KOMA class
  \KOMAoptions{parskip=half}}
\makeatother
\usepackage{xcolor}
\setlength{\emergencystretch}{3em} % prevent overfull lines
\setcounter{secnumdepth}{5}
% Make \paragraph and \subparagraph free-standing
\ifx\paragraph\undefined\else
  \let\oldparagraph\paragraph
  \renewcommand{\paragraph}[1]{\oldparagraph{#1}\mbox{}}
\fi
\ifx\subparagraph\undefined\else
  \let\oldsubparagraph\subparagraph
  \renewcommand{\subparagraph}[1]{\oldsubparagraph{#1}\mbox{}}
\fi


\providecommand{\tightlist}{%
  \setlength{\itemsep}{0pt}\setlength{\parskip}{0pt}}\usepackage{longtable,booktabs,array}
\usepackage{calc} % for calculating minipage widths
% Correct order of tables after \paragraph or \subparagraph
\usepackage{etoolbox}
\makeatletter
\patchcmd\longtable{\par}{\if@noskipsec\mbox{}\fi\par}{}{}
\makeatother
% Allow footnotes in longtable head/foot
\IfFileExists{footnotehyper.sty}{\usepackage{footnotehyper}}{\usepackage{footnote}}
\makesavenoteenv{longtable}
\usepackage{graphicx}
\makeatletter
\def\maxwidth{\ifdim\Gin@nat@width>\linewidth\linewidth\else\Gin@nat@width\fi}
\def\maxheight{\ifdim\Gin@nat@height>\textheight\textheight\else\Gin@nat@height\fi}
\makeatother
% Scale images if necessary, so that they will not overflow the page
% margins by default, and it is still possible to overwrite the defaults
% using explicit options in \includegraphics[width, height, ...]{}
\setkeys{Gin}{width=\maxwidth,height=\maxheight,keepaspectratio}
% Set default figure placement to htbp
\makeatletter
\def\fps@figure{htbp}
\makeatother
% definitions for citeproc citations
\NewDocumentCommand\citeproctext{}{}
\NewDocumentCommand\citeproc{mm}{%
  \begingroup\def\citeproctext{#2}\cite{#1}\endgroup}
\makeatletter
 % allow citations to break across lines
 \let\@cite@ofmt\@firstofone
 % avoid brackets around text for \cite:
 \def\@biblabel#1{}
 \def\@cite#1#2{{#1\if@tempswa , #2\fi}}
\makeatother
\newlength{\cslhangindent}
\setlength{\cslhangindent}{1.5em}
\newlength{\csllabelwidth}
\setlength{\csllabelwidth}{3em}
\newenvironment{CSLReferences}[2] % #1 hanging-indent, #2 entry-spacing
 {\begin{list}{}{%
  \setlength{\itemindent}{0pt}
  \setlength{\leftmargin}{0pt}
  \setlength{\parsep}{0pt}
  % turn on hanging indent if param 1 is 1
  \ifodd #1
   \setlength{\leftmargin}{\cslhangindent}
   \setlength{\itemindent}{-1\cslhangindent}
  \fi
  % set entry spacing
  \setlength{\itemsep}{#2\baselineskip}}}
 {\end{list}}
\usepackage{calc}
\newcommand{\CSLBlock}[1]{\hfill\break\parbox[t]{\linewidth}{\strut\ignorespaces#1\strut}}
\newcommand{\CSLLeftMargin}[1]{\parbox[t]{\csllabelwidth}{\strut#1\strut}}
\newcommand{\CSLRightInline}[1]{\parbox[t]{\linewidth - \csllabelwidth}{\strut#1\strut}}
\newcommand{\CSLIndent}[1]{\hspace{\cslhangindent}#1}

\KOMAoption{captions}{tableheading}
\makeatletter
\@ifpackageloaded{bookmark}{}{\usepackage{bookmark}}
\makeatother
\makeatletter
\@ifpackageloaded{caption}{}{\usepackage{caption}}
\AtBeginDocument{%
\ifdefined\contentsname
  \renewcommand*\contentsname{Table of contents}
\else
  \newcommand\contentsname{Table of contents}
\fi
\ifdefined\listfigurename
  \renewcommand*\listfigurename{List of Figures}
\else
  \newcommand\listfigurename{List of Figures}
\fi
\ifdefined\listtablename
  \renewcommand*\listtablename{List of Tables}
\else
  \newcommand\listtablename{List of Tables}
\fi
\ifdefined\figurename
  \renewcommand*\figurename{Figure}
\else
  \newcommand\figurename{Figure}
\fi
\ifdefined\tablename
  \renewcommand*\tablename{Table}
\else
  \newcommand\tablename{Table}
\fi
}
\@ifpackageloaded{float}{}{\usepackage{float}}
\floatstyle{ruled}
\@ifundefined{c@chapter}{\newfloat{codelisting}{h}{lop}}{\newfloat{codelisting}{h}{lop}[chapter]}
\floatname{codelisting}{Listing}
\newcommand*\listoflistings{\listof{codelisting}{List of Listings}}
\makeatother
\makeatletter
\makeatother
\makeatletter
\@ifpackageloaded{caption}{}{\usepackage{caption}}
\@ifpackageloaded{subcaption}{}{\usepackage{subcaption}}
\makeatother
\ifLuaTeX
  \usepackage{selnolig}  % disable illegal ligatures
\fi
\usepackage{bookmark}

\IfFileExists{xurl.sty}{\usepackage{xurl}}{} % add URL line breaks if available
\urlstyle{same} % disable monospaced font for URLs
\hypersetup{
  pdftitle={PL, NGP och Seqera},
  pdfauthor={Halfdan Rydbeck},
  colorlinks=true,
  linkcolor={blue},
  filecolor={Maroon},
  citecolor={Blue},
  urlcolor={Blue},
  pdfcreator={LaTeX via pandoc}}

\title{PL, NGP och Seqera}
\author{Halfdan Rydbeck}
\date{2024-10-24}

\begin{document}
\maketitle

\renewcommand*\contentsname{Table of contents}
{
\hypersetup{linkcolor=}
\setcounter{tocdepth}{2}
\tableofcontents
}
\bookmarksetup{startatroot}

\chapter*{Första sliden}\label{fuxf6rsta-sliden}
\addcontentsline{toc}{chapter}{Första sliden}

\markboth{Första sliden}{Första sliden}

Alterative titel: Bioinformatik på PL?

Nu skall vi prata om arbetsflödes hanteringsprogram och om hur de kan
komma till nytta för oss på Precisions medicinskt laboratorium

Den här boken är skriven med ``Quarto boo''k. This is a Quarto book. För
att lära mer om Quarto books kolla på
\url{https://quarto.org/docs/books}.

\section*{Blir många bilder}\label{blir-muxe5nga-bilder}
\addcontentsline{toc}{section}{Blir många bilder}

\markright{Blir många bilder}

Jag har ritat många diagram här fritt tagna ut fantasin. Eftersom det är
så många (nya) begrepp som presenteras så tror jag det underlättar om
man kan knyta ihop dem i mentala modellr. Mina diagram och bilder är ett
försök att underllätta den processen.

\section*{Svenska}\label{svenska}
\addcontentsline{toc}{section}{Svenska}

\markright{Svenska}

jag försöker göra presenationen på svenska. Det fungera bra när man
diskuterar organisationer etc men ibland finns mej veterligen inga bra
motsvarande svenska or. Då får vi slå över.

\section*{Useful links}\label{useful-links}
\addcontentsline{toc}{section}{Useful links}

\markright{Useful links}

https://alexd106.github.io/intro2R/Github\_intro.html\#Option\_2\_-\_RStudio\_first

\subsection*{Reg ostergotland Logos}\label{reg-ostergotland-logos}
\addcontentsline{toc}{subsection}{Reg ostergotland Logos}

\url{https://www.regionostergotland.se/ro/press/grafisk-profil/ladda-ner-logotyp}

\section*{Ploting Hex stickers}\label{ploting-hex-stickers}
\addcontentsline{toc}{section}{Ploting Hex stickers}

\markright{Ploting Hex stickers}

https://github.com/GuangchuangYu/hexSticker

Is there a package to plot multiple stickers??

\bookmarksetup{startatroot}

\chapter{Vad är PL?}\label{vad-uxe4r-pl}

Precisions medicinsk laboratorium (PL) i Linköping utför laboratorie
tjänster och analys för de \textbf{tre verksamhetsområdena} Klinisk
patologi, Klinisk genetik och Klinisk mikrobiologi.

\begin{itemize}
\item
  För Pat och klin\_gen så handlar det om att PL blir till-sänt prov och
  medföljande remiss och i bästa fall sänder tillbaka bakomliggande
  genetiska varianter.
\item
  För Mikro så handlar det om att man får bakterieprov och sänder
  tillbaka information om potentiella resistensgener och MLST varainter.
\end{itemize}

PL-Linköping utgör en av sju Genomic Medicine Centers (GMCs) i Genome
Medicine Sweden (GMS) som är ett samarbete mellan regioner med
universtietessjukvård som finansieras ungeför likvärdigt av: - Vinnova
(innovasionsmyndigheten) - universitetsjukhus regionerna

\section{PL-NGS}\label{pl-ngs}

Här ses PL isolerat utan de servade verskamhetsområdena.PL består av tre
enheter som vardera servar de tre versamhetsområdena. Hexagonen i kärnan
representerar det som de har gemensamt:

\begin{itemize}
\tightlist
\item
  Ledning
\item
  Sekvenseringsmaskiner

  \begin{itemize}
  \tightlist
  \item
    (Presentationen kommer att begränsa sig till att se på PL utifrån
    ett sekvenserings perspektiv, eftersm det är det som jag jobbar
    med.).
  \end{itemize}
\item
  Datornätverk
\end{itemize}

\bookmarksetup{startatroot}

\chapter{PL ingår i det nationella nätverket Genome Medicine Sweden
(GMS)}\label{pl-inguxe5r-i-det-nationella-nuxe4tverket-genome-medicine-sweden-gms}

Nätverket har sju noder, eller Genomisk Medicin Centrum (GMCs) som
utgörs ave regionerna universitetssjukhus.

På universitetssidan så har de ett systernätverk som organiseras av
SciLifeLab där noderna, eller Clinical Genomics (CG) key services,
ligger på sju av landets universitet. Det finns en vision om lokal
synergi mellan Clinical Genomcs och Genomiks Medicin Centra, där
meningen är att CGs skall stötta GMCs med kompetens.

GMS finansierar till hälften av projekt medlen som ansökts om hos
Vinnova och till hälfden av de involverade regionerna.

Det är framför allt två projekt som berör PL:

\begin{itemize}
\item
  Swelife
\item
  Systemdemonstrator \textbf{(NGP)}
\end{itemize}

\section{Data delning och central analys via NGP är del i ett större EU
sammanhang}\label{data-delning-och-central-analys-via-ngp-uxe4r-del-i-ett-stuxf6rre-eu-sammanhang}

Motsvarigheten till NGP på EU-nivå är förmodligen
\href{https://health.ec.europa.eu/ehealth-digital-health-and-care/european-health-data-space_en}{European
Health Data Space (EHDS)}. Drivande organistino bakom är {[}European
Healt
Union{]}(\url{https://commission.europa.eu/strategy-and-policy/priorities-2019-2024/promoting-our-european-way-life/european-health-union_en})

EHDS är viktigt för de är drivande i tolkning och tilläggslagstiftning
kring \href{https://gdpr-info.eu/}{General Data Protection Regulation}
(GDPR)

\subsection{}\label{section}

\bookmarksetup{startatroot}

\chapter{Vad är NGP?}\label{vad-uxe4r-ngp}

NGP är GMS's sätt bana väg för transformera dataflöden inom
svensk/nationell klinisk diagnostik och forskning.

\begin{itemize}
\tightlist
\item
  data strukturering
\item
  data standardisering
\item
  applikations utveckling
\item
  fördjupas samarbete med forsknigs infrastrukturen
\item
  Tillgängliggörande av data för akademin, myndigheter och näringsliv
\end{itemize}

Diagrammet visar att NGP utgörs av tree functionlaiteter som benämns NGP
repository (NGPr), NGP commander (NGPc) och NGP indexing (NGPi)

\section{Tre funktionaliteter}\label{tre-funktionaliteter}

\subsection{NGPr}\label{ngpr}

Det är hit det är tänkt att GMC-nodena skall kunna lasta upp sin data.
NGPc är ett fleranvändar system. Varje GMS nod är en användare eller
Tenant som det kallas i moln tjänst användbar språk. NGPC har ett data
lagrings segment som är privat för varja Tenant. I tillägg så har det
ett segment som är gemensamt.

\subsection{NGPi}\label{ngpi}

Detta är en funktinalitet som skall göra det lättare och snabbare att
hitta relevant data för olika användare av systemet.

\subsection{NGPc}\label{ngpc}

Har kommer analys verktyg att installeras som i huvudsak kommer att
utgöras av \textbf{pipelines eller workflows} eller bioinformatiska
arbetsflöden. I sin allra enklaste definitin så är ett sådan arbetsflöde
en antal program som kopplats ihop för att till tillsammans utföra
uppgifter vars komplexitet gör att det inte finns enskilda program som
kan utföra dem.

\section{Nytta för GMCerna}\label{nytta-fuxf6r-gmcerna}

\begin{itemize}
\tightlist
\item
  Tolknignsverktyg
\end{itemize}

\section{Övriga benefaktorer}\label{uxf6vriga-benefaktorer}

\begin{itemize}
\tightlist
\item
  Universitet - DDLS
\item
  Företag
\item
  Myndigheter, Fohm
\end{itemize}

\bookmarksetup{startatroot}

\chapter{Vad är ett bioinformatiskt
arbetsflöde?}\label{vad-uxe4r-ett-bioinformatiskt-arbetsfluxf6de}

\section{Exempel från Sällsynta diagnoser - Del i ett större
arbetsflöde}\label{exempel-fruxe5n-suxe4llsynta-diagnoser---del-i-ett-stuxf6rre-arbetsfluxf6de}

Här visas en schematisering av ett arbetsflöde på PL-rd.
Schematiskeringen är förenklad och framför allt i det bioinformatiska
flöden så har saker som tex identifiering av kopietalsförändringar
utelämnats.

\subsection{Manuell/fysisk del}\label{manuellfysisk-del}

\begin{enumerate}
\def\labelenumi{\arabic{enumi}.}
\item
  Det startar med att prov kommer in
\item
  DNA extraheras och bibliotek prepareras och lastas på en
  sekvenseringsmaksin
\end{enumerate}

\subsection{Bioinformatisk del}\label{bioinformatisk-del}

\begin{enumerate}
\def\labelenumi{\arabic{enumi}.}
\item
  Sekvenseringsmaskinen skriver sekvens och kvalitetsdata data,
  ursprungligen ofta i nått företagsspecifikt format.
\item
  Textfiler, sk fastq-filer, en för varje sekvenserad fragment, med
  sekvens och kvalitetsdata tillverkas.
\item
  Fragment-sekvens-datan filtreras och och alignas (radas upp) mot ett
  referensgenome
\item
  Varianter i förhållande till referens genomet identifieras och sparas
  i en Variant Call File VCF.
\item
  Varianterna annoteras med information som möjliggör efterkommande
  tolkning
\end{enumerate}

\subsection{Samarbete människa
maskin}\label{samarbete-muxe4nniska-maskin}

\begin{enumerate}
\def\labelenumi{\arabic{enumi}.}
\tightlist
\item
  Tolkning
\end{enumerate}

\section{Liknande arbetsflöden finns ju också för Patologi och
Mikrogrupperna}\label{liknande-arbetsfluxf6den-finns-ju-ocksuxe5-fuxf6r-patologi-och-mikrogrupperna}

Det visade arbetsflödet skall klara helgenoms analyser vilket gör
bioinformatik delen mer resurskrävande än de övriga flödena.

\section{}\label{section-1}

\bookmarksetup{startatroot}

\chapter{Förutom alla fördelar, som att stå på varandras axlar - så
finns det många utmaningar med bioinformatiska
arbetsflöden}\label{fuxf6rutom-alla-fuxf6rdelar-som-att-stuxe5-puxe5-varandras-axlar---suxe5-finns-det-muxe5nga-utmaningar-med-bioinformatiska-arbetsfluxf6den}

\section{Bioinformatiska arbetsflöden behöver nån form av beräknings
kraft fär att
köras.}\label{bioinformatiska-arbetsfluxf6den-behuxf6ver-nuxe5n-form-av-beruxe4knings-kraft-fuxe4r-att-kuxf6ras.}

\begin{itemize}
\tightlist
\item
  I huvudsak av effektiviseringssjäl/ekonomiska så sker det en stor
  omändring av av ekonomiska själ hur resurskrävande programmatiska
  uppgifter utförs. Konceptet dator och server håller på att bli
  förlegat ock ersätts av, HPC, molnberäkning och diverse andra begrepp.
  Bland annat hårdvaruutvecklingen ställer nya krav på mjukvara, och
  också på det området så föds kontinuerligt nya begrepp.
\end{itemize}

\section{Konkurenskraftig utveckling}\label{konkurenskraftig-utveckling}

\begin{itemize}
\item
  För att förbli konkurrenskraftigt så måste det finnas resurser för
  intensiv utveckling av flödet
\item
  Av ingående program och av flödet som sådant
\item
  Kräver ett system för att hantera versioner både av flödet och av de
  ingående komponenterna
\end{itemize}

\section{Skall kunna köras var som
helst}\label{skall-kunna-kuxf6ras-var-som-helst}

\begin{itemize}
\item
  Från tex forsknings- och hälso-perspektiv är det oftast önskvärt att
  analysresultaten är reproducerbara mellan labb.
\item
  Det kräver bland annat att flödet skall vara så flexibelt som möjligt
  med tanke på vilka hårdvaru-konfigurationer som det kan köras på.
\item
  resultaten skall bli desamma oavsett å vilket system som flödet körs
  på.
\end{itemize}

\section{Undvika att köra om delar där riktiga resultat redan genererats
- tar lång tid att
köra}\label{undvika-att-kuxf6ra-om-delar-duxe4r-riktiga-resultat-redan-genererats---tar-luxe5ng-tid-att-kuxf6ra}

\begin{itemize}
\item
  Därför svårt att felsöka
\item
  Tar lång tid att köra om vid stop
\end{itemize}

\section{Skall följa lagar och regler för den institution som använder
dem}\label{skall-fuxf6lja-lagar-och-regler-fuxf6r-den-institution-som-anvuxe4nder-dem}

\begin{itemize}
\tightlist
\item
  IVDR
\end{itemize}

\section{Variablelt behov av resurser från beräknings platformen
(runtime management); Kräver ett skalerbart
system}\label{variablelt-behov-av-resurser-fruxe5n-beruxe4knings-platformen-runtime-management-kruxe4ver-ett-skalerbart-system}

\begin{itemize}
\item
  RAM
\item
  Hårddiskutrymme
\item
  Parallell processering (kräver många processorer/cores)
\end{itemize}

\section{Olika flöden skall kunna köras samtidigt av flera olika
användare}\label{olika-fluxf6den-skall-kunna-kuxf6ras-samtidigt-av-flera-olika-anvuxe4ndare}

Här står alternativet mellan:

\begin{itemize}
\item
  En skalerbar fleranvändarmiljö
\item
  flera enskilda småskaliga miljöer, en för varje användare
\end{itemize}

\subsection{Mjukvarutveckling baserad på storskaligt samarbete och
system för
versionskontrollering}\label{mjukvarutveckling-baserad-puxe5-storskaligt-samarbete-och-system-fuxf6r-versionskontrollering}

\bookmarksetup{startatroot}

\chapter{Vad är Seqera?}\label{vad-uxe4r-seqera}

Based on Knowledge of high-throughput analysis and modern software
engineeering gained from building Nextflow the same people have

created a platform to make data-intensive research scalable, flexible,
and collaborative

Why do I mention a company the first thing after talking about
bioinformatics workflows? Because it has been growing out of free stuff
initaiteives, becasue it offers a way to deal with sensitive data and
because it offers a GUI interface to handling, running pipelines an
their output. I think its my best bet to make this presentaiton
interesting for you to show you (Emedgene endorsers) a a pipelein can be
run in seqera cloud.

\begin{itemize}
\tightlist
\item
  Seqera's affärs ide är att underlätta att arbeta med arbetsflöden
\item
  Grundarna ligger bakom två icke-kommersiella initiativ

  \begin{itemize}
  \tightlist
  \item
    Ett workflow hanterings verktyg som bygger på ett eget dedikerat
    programmeringsspråk, Nextflow
  \item
    Ett regel-set och socialt forum för uniform utvecklande av workflows
    - nf-core
  \end{itemize}
\end{itemize}

Seqera presents itself as an open science company. For me that is hard
to comprehend what it is. What dous that mean? They also say they want
to be a centrla hub at open science? Does open science want/need a hub
that is comercail? I dont knwo. In any case what

Seqera would not be the only compay that open science relies on. Github,
Docker and Singluarity are other examples. Undrstand their business
model is beyond my understande. What I can grasp though is theat they
provide interesting useful products for free.

There is also of course the micro array sequencing platfors tha tare
pillars of open genomics science. Many of the ENxtfow pipleline are in
direct competintion with Illumina products like Emedgene. For me the
most open poduct would win form any perspective,be it IVDR or usability.

\section{Many develpers of groupleaders, Nextflow and nf-core are
employess at
Seqera}\label{many-develpers-of-groupleaders-nextflow-and-nf-core-are-employess-at-seqera}

\section{Secrets, secure handling of sensitive
data}\label{secrets-secure-handling-of-sensitive-data}

\url{https://seqera.io/blog/pipeline-secrets-secure-handling-of-sensitive-information-in-tower/}

\subsection{Examples}\label{examples}

\url{https://aws.amazon.com/blogs/hpc/leveraging-seqera-platform-on-aws-batch-for-machine-learning-workflows-part-1-of-2/}

\section{Nextflow}\label{nextflow}

~solve problems with reproducible workflows

\bookmarksetup{startatroot}

\chapter{Hur kan liknande arbetsflöde(n) köras(på bästa sätt
)?}\label{hur-kan-liknande-arbetsfluxf6den-kuxf6raspuxe5-buxe4sta-suxe4tt}

I dagsläget så verkar det som att det i flesta fall blir mest effektivt
att använda sig av en skalerbar fleranvändarmiljö, dvs ett cluster eller
en moln miljö.

I vårt fall när det skall tas hänsyn till säkerhetsklassad/känslig data
så krävs antingen lokalt system eller ett system med säker inloggning.

\section{Nya lösningar inom både hardvaru- och mjukvaru-design gör det
allt lättare att köra men också utvecka och vidarutveckla
arbetsföden}\label{nya-luxf6sningar-inom-buxe5de-hardvaru--och-mjukvaru-design-guxf6r-det-allt-luxe4ttare-att-kuxf6ra-men-ocksuxe5-utvecka-och-vidarutveckla-arbetsfuxf6den}

\subsection{Hårdvara
+Operativesystem/hårdvaruhateringssystem}\label{huxe5rdvara-operativesystemhuxe5rdvaruhateringssystem}

\subsubsection{HPC}\label{hpc}

\begin{itemize}
\item
  NGP (altair grid engine)
\item
  PDC-Dardel (slurm)
\end{itemize}

\subsubsection{Moln}\label{moln}

\begin{itemize}
\tightlist
\item
  AWS
\item
  Azure
\end{itemize}

\subsubsection{Lokal server}\label{lokal-server}

\begin{itemize}
\tightlist
\item
  Blir svårhanterligt\ldots{}
\end{itemize}

\subsection{Beräkningsmiljöer}\label{beruxe4kningsmiljuxf6er}

\begin{itemize}
\tightlist
\item
  Altair Grid engine
\item
  Slurm
\end{itemize}

\subsection{Mjukvara -
hanteringsprogram}\label{mjukvara---hanteringsprogram}

\begin{itemize}
\item
  Version
\item
  container
\item
  arbetsflöde
\end{itemize}

\bookmarksetup{startatroot}

\chapter{Hårdvara och
beräkningsmiljöer}\label{huxe5rdvara-och-beruxe4kningsmiljuxf6er}

Vi ser på detta utfrån arbetsflödeshanterings program perspektiv. Dessa
program behöver ha gränssnitt kan samarbeta med de olika typeran av
hårdvara

Viktigt kriteria är att att det skall gå att lagra och analysera
sensitiv klinisk patient data. Det finns ett EU belsut om att det skall
vara lova att lagara och analysera den typen av data på molnlösningar
(med hårdvra i Europa?).

Det gör att vi i nuläget inte kan dra igång med molntjänster.

\section{HPC}\label{hpc-1}

Ett HPC består av en grupp datorer/servrar(noder) som kan samarbeta för
att utföra en gemensam beräknings uppgift. Varje nod i tar emot och
processerar beräknings uppgifter oberoende av varandra. Noderna
koordinerar och synkroniserar uppgifterna för att till slut producerar
ett sammanslaget resultat.

Även om arkitekturen hos ett super computing cluster eller HPC kan vara
no så komplicerad och skilja sig mycket från en generation till nästa så
är det inte nödvändigt för den generella användaren av systemet. Det
viktiga är veta processen för hur man skickar iväg ett jobb som ansöker
om beräkningsnoder (fysiska grupperingar av processorer) som gör de
beräkningar som man är intresserade av. Ett HPC är byggt för att ha
flera användare så därför kommer det alltid med en mjukavara som kan
sätta sätat upp köer för användare.

Det finns en handfull mjuk-varor för kö-hantering som dominerar
användarmarknaden. Några är gratis och några kostar pengar.

Det som är genomgående för mjukvaran är att de erbjuder möjligheten att
formulera skript som definierar:

\begin{itemize}
\tightlist
\item
  Hur mycket datorkraft man vill ta från HPCn
\end{itemize}

\subsection{Komponenter}\label{komponenter}

\subsubsection{Beräkning}\label{beruxe4kning}

\begin{itemize}
\item
  Head eller login node
\item
  Vanliga beräknings noder

  \begin{itemize}
  \tightlist
  \item
    CPU, GPU
  \end{itemize}
\end{itemize}

\subsubsection{Lagring}\label{lagring}

\begin{itemize}
\item
  Fysisk lagring (on premise). Kan ofta vara överlägser ur
  hsitghets/performacne perspektiv. Tillåter parallella filsystem oc
  hlow latancy access.
\item
  Moln baserad lagring. Skalerbart. Tillåter ofta hög hastighet(numera).
\item
  Hybrid
\end{itemize}

\subsubsection{Nätverk}\label{nuxe4tverk}

Noderna måste kunna kommunicera med varann. Viktigt är att upnå högst
möjlgia hastighet.

\subsubsection{HPC Jobbschemaläggare och
resurshanterare}\label{hpc-jobbschemaluxe4ggare-och-resurshanterare}

\begin{itemize}
\tightlist
\item
  Vikitg komponent hos HPC
\end{itemize}

\subsection{Vanliga typer av
arkitekturer}\label{vanliga-typer-av-arkitekturer}

Paralllel, cluster och grid beräkning

En HPC desing can kombinera Paralllel, cluster och grid design alltså
innefatta alal tre.

\subsubsection{Parallell beräkning}\label{parallell-beruxe4kning}

Förmåga att distribuera en beräknings uppgift eller data på flera
noder/processorer

\subsubsection{Cluster beräkning}\label{cluster-beruxe4kning}

Koppla ihop flera datorer till en enhet

\subsubsection{Grid och distribuerar
beräkning}\label{grid-och-distribuerar-beruxe4kning}

Handlar om att koppla ihop geografiskt spridda beräkngns resurser till
en virtuell enhet. Så til skillnad från ett cluster så involverar en
grid enheter från flera olika platser och organisationer.

\subsection{NGP (äger vår data
själv)}\label{ngp-uxe4ger-vuxe5r-data-sjuxe4lv}

Det finns inte så mycket dokumentation för NGP från ett användar
perspektiv/ Ingan användamanual. Där för kan vi ta ett annat svensk HPC
upsätt för icke sensitive data.

\subsubsection{Lokalt HPC, grid
computing}\label{lokalt-hpc-grid-computing}

Det kallas grid computing för att det kommersiella köhantinrings
programmet heter Altaier GRID engine.

Plan på att installera ett lokalt HPC. De kommer att likna NGP i sin
uppsättning

\subsection{PDC-Dardel}\label{pdc-dardel}

\section{Moln/internet beräkning}\label{molninternet-beruxe4kning}

Tillgång till servrar, lagring, database, nätverk och mjukvara,
analysverktyg och intelligens.

\subsection{\texorpdfstring{\href{https://www.researchgate.net/publication/230800624_Cloud_intelligence_what_is_REALLY_new}{Intelligence}}{Intelligence}}\label{intelligence}

\begin{itemize}
\tightlist
\item
  Elasticitet i tillgång på beräkningsresurser. Förmågan att dynamiskt
  dra in nya data resurser
\item
  Datorstödd affärsanalys (analytiska processer som undersöker data och
  presetnerar anvädnbar information baserade på till exemple rapporter)
\item
  Ofta/alltid affärsdrivande och kräver en betalnings modell. Enskilda
  användarkonton där var och en betalar för sig.
\item
  Oerhört centraliserad resurs med stor kompentens (AWS/Azure
  vs.~NAISS/GMS)
\end{itemize}

\subsection{Wikipedia}\label{wikipedia}

Moln beräkning innefattar så många olika saker att en definiton riskear
att bli vag.

\begin{itemize}
\item
  On demand, självbetjäning
\item
  Tillgänglig för alla sortes enheter,. Mobi, surfplattor och
  arbetsstationer.
\item
  Snabb elasticitet
\item
  Övervakad resursanvändning \#\#\#\# Påstådda fördelar
\item
  Kostnade. Betalar bara för när resurserna används
\item
  Webgränssnitt gär att man kopal upp sig med vad som helst som har en
  web browser
\item
  Inget on-premise underhåll
\end{itemize}

\subsubsection{Tänkbara nackdelar}\label{tuxe4nkbara-nackdelar}

\begin{itemize}
\tightlist
\item
  data säkerhet. Moln användare anförtror sin data till tredjeparts
  leverantörer.
\item
  Reducerad transparans. kan sakan full översikt/insikt i hur resurser
  övervakas och rapporterass
\item
  Fullstädigt översikt över hur systement fungara kan bli omöjligt.Något
  som kan utläsas av metaforen ``moln''.
\item
  Migratin från moln kan vara komplicerat
\item
  Impementering av mjukvara och arbetsflöden kan drabbas av problem som
  har att göra med bl.a. distriburad beräkningskapacitet.
\item
  Om man inte håller bra koll på vilak resurser som körs och vad de
  kostar så kan man få en överaskning i form av kostander.
\end{itemize}

\subsubsection{FIgur text}\label{figur-text}

\paragraph{Infrastructure as a service
(IaaS)}\label{infrastructure-as-a-service-iaas}

Tex svervrar, Lagringsdiskar och nätverk

EC2 is and IaaS

\paragraph{Platform as a service
(PaaS)}\label{platform-as-a-service-paas}

Tex operativ system, databaser, säkerhetsprogram

\paragraph{Software as a service
(SaaS)}\label{software-as-a-service-saas}

Enskilda program. I assume this is when Netflix, AirBnB are examples of
Saas on AWS

Famous Saas companies:

Adobe Zoom Microsoft

\subsection{\texorpdfstring{\href{https://www.spiceworks.com/tech/cloud/articles/aws-basics/}{AWS}}{AWS}}\label{aws}

On demand molnberäkningslösning som inkluderar 200+ tjänster, platformar
och APIs som används av företag, myndigheter och privatpersonen som alla
betalar efter de resurser de använder.

Elasitic Comput (EC2), Amazon's virtualla beräknings service, Glacier,
en lågpris moln lagrings service, och S3, Amazon's lagrings system, är
tre grund componenter i AWS.

\section{Lokal server}\label{lokal-server-1}

\bookmarksetup{startatroot}

\chapter{Jobbschemaläggare och
resurshanterare}\label{jobbschemaluxe4ggare-och-resurshanterare}

\section{Gratis}\label{gratis}

\subsection{Slurm}\label{slurm}

SLURMs primära funktion är att allokera resurser inom klustret till dess
användare . Resurshantering kan innefatta hantering av noder, sockets,
kärnor och hypertrådar. Dessutom kan resursallokering baserad på
topologi, mjukvarulicenser och generiska resurser som GPU:er hanteras av
SLURM

\url{https://slurm.schedmd.com/documentation.html}

\url{https://blogs.oracle.com/research/post/a-beginners-guide-to-slurm?fireglass_rsn=true\#fireglass_params&tabid=a133ec835f7b2014&start_with_session_counter=2&application_server_address=sg-integration2-europe-west3.prod.fire.glass}

\subsubsection{Submitta ett jobb}\label{submitta-ett-jobb}

You can submit a job script to the Slurm queue system from the login
node with:

\begin{verbatim}
sbatch mitt_slurm_jobb.sh
\end{verbatim}

mitt\_slurm\_jobb.sh

\begin{verbatim}
#!/bin/bash -l
# The -l above is required to get the full environment with modules

# Set the allocation to be charged for this job
# not required if you have set a default allocation
#SBATCH -A naissYYYY-X-XX

# The name of the script is myjob
#SBATCH -J myjob

# The partition
#SBATCH -p main

# 10 hours wall-clock time will be given to this job
#SBATCH -t 10:00:00

# Number of nodes
#SBATCH --nodes=4

# Number of MPI processes per node
#SBATCH --ntasks-per-node=128

# Run the executable named myexe
# and write the output into my_output_file
srun ./myexe > my_output_file
\end{verbatim}

Running Bowtie

\begin{verbatim}
#!/bin/bash
#SBATCH --job-name=bowtie2_example
#SBATCH --cpus-per-task=8
#SBATCH --time=00:10:00
#SBATCH -o Bowtie_test.o%j
#SBATCH --partition=standard
#SBATCH --account=<YOUR_ALLOCATION>

#Load the Bowtie Module
module load gcc
module load bowtie2

# Change to temp working directory with example files
cd /scratch/$USER/bowtie_temp

# Indexing a reference genome
bowtie2-build ./example/reference/lambda_virus.fa lambda_virus

# Aligning example reads, standard example
bowtie2 -p $SLURM_CPUS_PER_TASK -x lambda_virus -U ./example/reads/reads_1.fq -S align.sam

# Paired-end example
bowtie2 -p $SLURM_CPUS_PER_TASK -x lambda_virus -1 ./example/reads/reads_1.fq -2 ./example/reads/reads_2.fq -S align2.sam

# Local alignment example
bowtie2 -p $SLURM_CPUS_PER_TASK --local -x lambda_virus -U ./example/reads/longreads.fq -S align3.sam
\end{verbatim}

\section{Kommersiell}\label{kommersiell}

\subsection{Altair Gridengine}\label{altair-gridengine}

\subsubsection{Submitta ett jobb}\label{submitta-ett-jobb-1}

\begin{verbatim}
qsub -V -b n -cwd mitt_gridengine_jobb.sh
\end{verbatim}

mitt\_gridengine\_jobb.sh

\begin{verbatim}
#!/bin/bash
                    #$ -N run_bowtie2
                    #$ -cwd
                    #$ -pe smp 6
                    #$ -l h_vmem=6G


                    infile=/data/bioinfo/READS2/R1_001.fastq.gz
                    outfile=/data/bioinfo/READS2/aln/R1_001.sam
                    btindex=/data/bioinfo/genome_data/Caenorhabditis_elegans/UCSC/ce10/Sequence/BowtieIndex/genome

                    gzip -dc $infile | bowtie  --chunkmbs 300 --best -m 1 -p 6 --phred33 -q $btindex   -  -S $outfile
\end{verbatim}

\bookmarksetup{startatroot}

\chapter{Vad kan
versionskontrolleras?}\label{vad-kan-versionskontrolleras}

Varje nf-core arbetsflöde görs tillgängligt via Github en website med
det huvudsakliga syftet att dela versionskontrollerade project över
internet. Versionkontrollering är användbart för nästan allt skapande
som kan utföras på en dator. Här pratar vi om det utifrån perspekivet
att arbetsflödeshanterings program

\begin{itemize}
\item
  En fil
\item
  En folder med filer och subfoldrar
\item
  Den här presentationen
\item
  Ett arbetsfödesschema
\end{itemize}

\section{Det allra enkaste exemplet - en
fil}\label{det-allra-enkaste-exemplet---en-fil}

Som en textfil, eller en program fil.

Istället för att spara filen som synliga odokumenterade versioner.Tex
Version\_1,version2,final\_version\_3 Så använder man ett versions
hanterings program som låter oss spara versioner i en dold
databas/repository döljer och ber oss som att dokumentera ändringarna i
varje version. Med verktyget såkan vi och vilja vilka ändringar som
gjort som skall följa med in i ett version shot.

\section{En folder}\label{en-folder}

Smam sak kan göras simultatnt för flera filer i en foldr

\section{Den här presentationen}\label{den-huxe4r-presentationen}

den här presentationen har utveckalts med versins hanterigns progam

\section{Ett arbetsflödes schema}\label{ett-arbetsfluxf6des-schema}

Programmatiska aarbetsflöden utveckal under versins kontroll

\bookmarksetup{startatroot}

\chapter{git, ett system för
versionskontrollering}\label{git-ett-system-fuxf6r-versionskontrollering}

\section{Vad kan
versionkontrolleras?}\label{vad-kan-versionkontrolleras}

\begin{itemize}
\item
  en fil
\item
  En folder (-strukture) med flera filer
\end{itemize}

\section{git}\label{git}

Git bygger på att man har en folder (workspace, working directory,
project folder) som innehåller ett ett flera dokument som man vet att
man kommer att vidareutveckla över tid med viss osäkerhet, där man kan
ångra sig och därför behöva återgå till tidigare versioner. Genom att
gömma alla versioner utom den som man jobbar på för tillfället så
minimerar programmet den distraherande påverkan som det kan ha att se
flera versioer av sitt/sina dokument samtidigt.

Eftersom vidarutvecklingen sker över tid så samarbetar man alltid
åtminstone med sitt framtida jag, tex genom att dokumentera/sammanfatta
ändringarna som gjorts i varje version. Det underlättar att vid ett
senare tillfälle senare bestämma vilken version man vill återgå till.
Git underlättar också samarbeta med andra tex genom att detektera när
två filändringar överlappar/är motstridigar och erbjuda verktyg för att
editera/slå ihop de motstridiga ändringarna.

Efter att ha installera programmet så börjar arbetet med
versionkontollering genom att initiera en s.k. repository i
föräldrafoldern för dina dokument.

\begin{verbatim}
git init
\end{verbatim}

Kommandot skapat en gömd folder ``.git'' i föräldrafolder. I den gömda
foldern så kommer dokumenterade ögonblicksbilder av din/dina fil/filer
att sparas.

Arbetsgången med git innnhåller tre centrala moment som motsvars av tre
filtillstånd:

\begin{enumerate}
\def\labelenumi{\arabic{enumi}.}
\item
  Gör filändringar; filen/filerna är ändrad(e)
\item
  Välj ändringar som skall förevigas i en ögonblicksbild (med commandot
  git add); Filens/filernas ändringar har valts ut till nästa
  ögonblicksbild; staged
\item
  Spara utvallda ändringar i som en ögnblickbild i
  .git-databsen/repositoryn tilllsammans med dokumentation (med
  commandot git commit -m); Filens ändringar har sparats i en
  ögonblicksbild; commited
\end{enumerate}

Överkurs är sedan att lära sig hur man kan gå tillbaka till (checka ut)
tidigare versioner.

\section{Några egenskaper hos git}\label{nuxe5gra-egenskaper-hos-git}

\subsection{Tillåter icke linjärt
skapande}\label{tilluxe5ter-icke-linjuxe4rt-skapande}

Dvs att man kan jobba på olika delar samtidigt. Du jobbar på en fil som
berör ett visst samnahang av det du vill säga med ditt projekt. Så får
du en idee om nått du vill säga om ett helt annat samanhang. Då kan du
spara det du höll på med börja jobab med en nya ideen och sedan problem
fritt återgå til det du höll på med.

\subsection{Simultan/paralllel
utveckling}\label{simultanparalllel-utveckling}

Flera kan samarbeta på ett dokument utan att riskera att skiva över
varandras bidrag.

\subsection{Gör det väldigt svårt att förlora material som en gång
sparats in en
version.}\label{guxf6r-det-vuxe4ldigt-svuxe5rt-att-fuxf6rlora-material-som-en-guxe5ng-sparats-in-en-version.}

\section{Github}\label{github}

Github är en website med den huvudsakliga funktionen att dela
versionskontrollerade projekt över internet.

\bookmarksetup{startatroot}

\chapter{Kollektivt versionskontrollerad mjukvaruutveckling
A}\label{kollektivt-versionskontrollerad-mjukvaruutveckling-a}

Molnbaserat Git kodförråd (repository). Det gör det lättare för personer
och grupper att använda GIt för versionskontrollering och samarbete.

Github gränssnittet är användarvänligt nog så att tom nyblivna
programmerare (ni) kan använda det. Github är så användarvänligt att en
del personer tom använder det för att skriva böcker eller PhD uppsatser.

Vem som helst kan skapa ett konto, logga in och lasta upp versions
kontrollerade kodförråd/dokumet foldrar.

\section{Arbetsfolder}\label{arbetsfolder}

Dett är platsen där du jobbabr för tillfället, där dina filer håller
till. Den platsen kallas också ``untracked'' område hos/av git.
Filändringar kommer att markeras och bli sedda i
arbetsträdet/arbetsfoldern. Om du gör filändringar här utan vidare
tilltag och sedan raderar/skriver över ändringarna så kommer det att
vara förlorade. Dette eftersom ändringarna ännu inte sagt att git skall
bry sig om ändringaran. Om man gr ändirngar där såk ommer git att se
dem, men inte förren git blir tillsgt att ``Hej, föj de här filerna
ändringar'', kommer git att spara nått som sker med dem.

\section{Staging area/index/förberdelse
område/fil}\label{staging-areaindexfuxf6rberdelse-omruxe5defil}

The staging area är en fil i Git foldern som sparar information om vad
som tas med i nästa ``commit''/ögonblicksbild.

\section{Lokalt kodförråd/.git foldern/ Git
foldern}\label{lokalt-kodfuxf6rruxe5d.git-foldern-git-foldern}

Det är här sm GIt spara metadatan och objectdatabasen för projektet. Git
foldern är den som lasta upp till github och som sedna kan kopieras när
man klanr från en annan dataor.

\section{Github/fjärr kodförråd}\label{githubfjuxe4rr-kodfuxf6rruxe5d}

Det här är . git foldern som lastats upp till Github.

\bookmarksetup{startatroot}

\chapter{Kollektivt versionskontrollerad mjukvaruutveckling
B}\label{kollektivt-versionskontrollerad-mjukvaruutveckling-b}

\bookmarksetup{startatroot}

\chapter{Ikoner som kommer användas för versionskontrollerande
enheterna}\label{ikoner-som-kommer-anvuxe4ndas-fuxf6r-versionskontrollerande-enheterna}

\bookmarksetup{startatroot}

\chapter{Program utvecklade i en specifik hårdvarukonfiguration skall
kunna köras i vilken som helst
annan.}\label{program-utvecklade-i-en-specifik-huxe5rdvarukonfiguration-skall-kunna-kuxf6ras-i-vilken-som-helst-annan.}

\section{Vad är software
dependancies}\label{vad-uxe4r-software-dependancies}

En program fil som skrivs fungerar alltid i en kontext av andra program
filer och för att fungera som tänkt. Programeringsspråk baserar sig ofta
på kodbibliotek, ``libraires'' som kan referereas i en programfil för
att användas för sin dedikerade uppgift. Det finns tex kodbibliotek för
att hämta dta ur en databas. Då kan en mjukvara beroende av det
kodbilboteket för att kommunicera med databaser för att fungera som
tänkt.

Ett software dependancy är en kodbibliotek eller paket som återanvänds i
ett mjukvara. Tex så kan ett maskinlärningsprojekt anropa en pythn
bibliotek för att bygga modeller.

\subsection{Conda packages (paket)}\label{conda-packages-paket}

\section{Container hanterings
program}\label{container-hanterings-program}

\subsection{Singularity}\label{singularity}

\subsection{Docker}\label{docker}

\bookmarksetup{startatroot}

\chapter{Container hanteringsprogram och container
register}\label{container-hanteringsprogram-och-container-register}

\bookmarksetup{startatroot}

\chapter{Arbetsflödehanteringsprogram, Nextflow, ett av
många}\label{arbetsfluxf6dehanteringsprogram-nextflow-ett-av-muxe5nga}

\section{}\label{section-2}

Nextflow är en gratis och open-source programvara, som utvecklas av
företaget Seqera labs.

Nextflow har/är ett eget dedikerat programmeringsspråk, ett s.k. Domain
Specific Language (DSL) som tillvärkningen av pipelines baserar sig på.

Språket är tillverkat baserat på samma idee som Linux baserar sig på.
Anvädn små kraftfulla kommondolinjebaserade program som när de länkas
samma underlättar utförandet av komplexa datahanterings uppgifter.

\section{Funktionskraven hos ett arbetsflödes hanteringsprogram omfattar
mer än bara att knyta ihop ett antal program till ett
pärlband.}\label{funktionskraven-hos-ett-arbetsfluxf6des-hanteringsprogram-omfattar-mer-uxe4n-bara-att-knyta-ihop-ett-antal-program-till-ett-puxe4rlband.}

det omfattar dessutom

\begin{itemize}
\item
  skalerbarhet
\item
  reproducerbarhet
\item
  förmåga att integrerar mjukvarupaket, programmiljö hanteringsprogram
  som Docker, Singluarity och Conda för att möjligöra att sammankoppla
  scriptspråk såsom BASH, R och Python.
\item
  Förenkla att köra pipelinen på olik platformar som tex cloud eller
  HPC-baserade infrastrukturer
\item
  hantering mha tex social media av en friviligbaserade intressegrupp
  för utveckling av standarder för och arbetfödena i sig själva som är
  av gemensamt intresse. Användar och utvecklare socialisear i ett stort
  virrvarr av olika intressen. Slack, github; Seqera??
\end{itemize}

Om ni frågar mig så är det förvirrande att Nextflow används som ett
paraply begrepp för flera funktioner.

\section{Nextflow är}\label{nextflow-uxe4r}

\subsection{Ett scriptspråk dedikerat för att bygga programmatikska
arbetsföden}\label{ett-scriptspruxe5k-dedikerat-fuxf6r-att-bygga-programmatikska-arbetsfuxf6den}

\subsection{Ett vertyg för att interagera
med}\label{ett-vertyg-fuxf6r-att-interagera-med}

\subsubsection{container
hanteringsprogram}\label{container-hanteringsprogram}

\subsubsection{versions
hanteringsprogram}\label{versions-hanteringsprogram}

\subsubsection{Executors/platforms}\label{executorsplatforms}

Kan kommunicera med dessa och specificera vilka resurser varje enskil
modul kräver.

\subsubsection{}\label{section-3}

\bookmarksetup{startatroot}

\chapter{Arbetsflödet och programmen som det omfattar är
versionskontrollerade}\label{arbetsfluxf6det-och-programmen-som-det-omfattar-uxe4r-versionskontrollerade}

\bookmarksetup{startatroot}

\chapter{Skalbarhet}\label{skalbarhet}

\bookmarksetup{startatroot}

\chapter{Nextflow arbetsflöden som följer och inte följer
nf-core}\label{nextflow-arbetsfluxf6den-som-fuxf6ljer-och-inte-fuxf6ljer-nf-core}

\bookmarksetup{startatroot}

\chapter{Behov av kontinuerlig utveckling av
algoritmer}\label{behov-av-kontinuerlig-utveckling-av-algoritmer}

\bookmarksetup{startatroot}

\chapter{Konfigurera och köra på från kommando
linjen}\label{konfigurera-och-kuxf6ra-puxe5-fruxe5n-kommando-linjen}

\bookmarksetup{startatroot}

\chapter{Om Seqera igen}\label{om-seqera-igen}

\bookmarksetup{startatroot}

\chapter{Konfigurerar och köra på
Seqera}\label{konfigurerar-och-kuxf6ra-puxe5-seqera}

\bookmarksetup{startatroot}

\chapter*{References}\label{references}
\addcontentsline{toc}{chapter}{References}

\markboth{References}{References}

\phantomsection\label{refs}
\begin{CSLReferences}{0}{1}
\end{CSLReferences}



\end{document}
